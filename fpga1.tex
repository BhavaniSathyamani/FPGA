\documentclass{beamer}
\usepackage{graphicx}
\usetheme{Boadilla}
\usecolortheme{beaver}
\title{Floating Point Adder}
\author{EE18ACMTECH11006    Bhavani Machnoori   \\ 
EE18MTECH11005   Ajay Hase}

\date{\today}

\begin{document}
\begin{frame}
\titlepage
\end{frame}

\begin{frame}
\includegraphics[scale=0.5]{table1.png}

\textbf{Steps for converting decimal to floating point number:} \\

1. Convert a Decimal number to Binary number $(975.75)_10$ = $(1111001111.11)_2 $ \\
2. Normalize the number 1.11100111111* $2^9$\\
3. From this normalized number we can fill all 32-bits of floating point number
Sign bit = 0 (number is positive) \\
4. Exponent = Bias + 9 = 127 + 9 = $(136)_10$ = $(1000 1000)_2$ \\
5. Fraction part will contain all the bits after decimal point. \\
6. $(975.75)_10$ is expressed as shown below in single precision floating point format.\\
\includegraphics[scale=0.45]{table2.png}
\end{frame}
\begin{frame}
Steps for floating point adder:\\
1.Sort:Find the largest number.\\
2.Align:Make the exponent equal.\\
3.Add/Sub:Perform addition or subtraction\\
4:Normalize: set MSB of mantissa  
\end{frame}
\begin{frame}
Examples:

\includegraphics[scale=0.7]{Untitled11.png}
\end{frame}

\begin{frame}


Flow chart : 
\includegraphics[scale=0.45]{algorithm.png}

\end{frame}
\begin{frame}
\textbf{Applications:
}  \\

Floating point numbers are used in various applications such as medical
imaging, radar, telecommunications:\\
1. CPU \\
2. Calculators
\end{frame}

\end{document}        